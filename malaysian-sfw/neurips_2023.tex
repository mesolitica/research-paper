\documentclass[preprint]{article}


% if you need to pass options to natbib, use, e.g.:
%     \PassOptionsToPackage{numbers, compress}{natbib}
% before loading neurips_2023


% ready for submission
\usepackage{neurips_2023}


% to compile a preprint version, e.g., for submission to arXiv, add add the
% [preprint] option:
%     \usepackage[preprint]{neurips_2023}


% to compile a camera-ready version, add the [final] option, e.g.:
%     \usepackage[final]{neurips_2023}


% to avoid loading the natbib package, add option nonatbib:
%    \usepackage[nonatbib]{neurips_2023}


\usepackage[utf8]{inputenc} % allow utf-8 input
\usepackage[T1]{fontenc}    % use 8-bit T1 fonts
\usepackage{hyperref}       % hyperlinks
\usepackage{url}            % simple URL typesetting
\usepackage{booktabs}       % professional-quality tables
\usepackage{amsfonts}       % blackboard math symbols
\usepackage{nicefrac}       % compact symbols for 1/2, etc.
\usepackage{microtype}      % microtypography
\usepackage{xcolor}         % colors
\usepackage{listings}
\usepackage{cite}
\usepackage{graphicx}
\usepackage{hyperref}
\usepackage{amsmath}


\lstset{
  breaklines=true,
  basicstyle=\ttfamily\small,
  commentstyle=\color{green!40!black},
  keywordstyle=\color{blue},
  numberstyle=\tiny\color{gray},
  numbers=none,
  frame=single,
  breaklines=true,
  breakatwhitespace=true,
  captionpos=b,
  showstringspaces=false,
  columns=fullflexible,
}

\hypersetup{
    pdftitle={Adapting Safe-for-Work Classifier for Malaysian Language Text: Enhancing Alignment in LLM-Ops Framework},
    pdfauthor={Aisyah Razak, Ariff Nazhan, Ariff Nazhan},
    pdfsubject={Natural Language Processing, Large Language Models, Machine Learning},
    pdfkeywords={language models, malay natural language processing, deep learning},
    colorlinks=true,
    linkcolor=blue,
    citecolor=blue,
    urlcolor=blue
}
\title{Adapting Safe-for-Work Classifier for Malaysian Language Text: Enhancing Alignment in LLM-Ops Framework}
\author{
  Aisyah Razak\thanks{aisyahrazak171@gmail.com} \and
  Ariff Nazhan\thanks{ariffnzhn@gmail.com} \and
}

\begin{document}

\maketitle

\begin{abstract}

\end{abstract}

\section{Introduction}

\section{Data Source}
The data for this study was collected from various platforms, including social media, public forums, and publicly available datasets. The majority of the data is in the malay language and relevant to the malay context.
\subsection{Social Media}
Data was collected from popular social media platforms such as Twitter and Facebook. Two main approaches were employed to gather relevant data:

1. Keyword-based scraping: a list of keywords associated with explicit content was compiled. These keywords were used to extract tweets from the platform.

2. Profile-based scraping: a list of profiles known for regularly posting NSFW content was curated. Posts from these profiles were then scraped to obtain a more targeted dataset.

The combination of these two scraping methods resulted in a comprehensive and diverse dataset from social media, capturing both keyword-specific content and data from profiles that frequently share explicit material.


\subsection{Public Dataset}

\section{Finetuning Procedure}

\section{Evaluate}

\section{Acknowledgement}

Special thanks to Malaysia-AI volunteers especially \href{https://www.linkedin.com/in/ammar-azman/}{Ammar Azman}, \href{https://www.linkedin.com/in/amzar96/}{M. Amzar}, \href{https://www.linkedin.com/in/muhammad-farhan-helmy-0529501a7/}{Muhammad Farhan}, \href{https://www.linkedin.com/in/syafie-nizam/}{Syafie Nizam}, \href{https://www.linkedin.com/in/alif-aiman-1b334b24b/}{Alif Aiman}, \href{https://www.linkedin.com/in/azwan-zuharimi/}{Azwan Zuharimi} and \href{https://www.linkedin.com/in/haziqzikry/}{Haziq Zikry} for contributing dataset to train Malaysian Reranker models.

We would like to express our gratitude to NVIDIA Inception for generously providing us with the opportunity to train our model on the Azure cloud. Their support has played a crucial role in the success of our research, enabling us to leverage advanced technologies and computational resources.

We extend our thanks to the wider research community for their valuable insights and collaborative discussions, which have greatly influenced our work. This paper reflects the collective efforts and contributions from both NVIDIA Inception and the broader research community.

\section{Conclusion}

\bibliography{neurips_2023}{}
\bibliographystyle{unsrt}

\end{document}